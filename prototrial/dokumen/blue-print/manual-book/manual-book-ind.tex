% ---
% Packages
% ---
\documentclass[11pt,a4paper,twoside,draft,onecolumn]{book}
\usepackage[utf8]{inputenc}
\usepackage[strict]{changepage}
\usepackage{xcolor}
\usepackage{framed}
\usepackage{fourier}
\usepackage{graphicx}
\usepackage{geometry}
\usepackage{layout}
\usepackage{pifont}
\usepackage{fontawesome}

\definecolor{formalshade}{rgb}{0.95,0.95,1}

\newenvironment{formalred}{%
	\def\FrameCommand{%
		\hspace{1pt}%
		{\color{red}\vrule width 2pt}%
		{\color{formalshade}\vrule width 4pt}%
		\colorbox{formalshade}%
	}%
	\MakeFramed{\advance\hsize-\width\FrameRestore}%
	\noindent\hspace{-4.55pt}% disable indenting first paragraph
	\begin{adjustwidth}{}{7pt}%
		\vspace{2pt}\vspace{2pt}%
	}
	{%
		\vspace{2pt}\end{adjustwidth}\endMakeFramed%
}

\newenvironment{formalblue}{%
	\def\FrameCommand{%
		\hspace{1pt}%
		{\color{blue}\vrule width 2pt}%
		{\color{formalshade}\vrule width 4pt}%
		\colorbox{formalshade}%
	}%
	\MakeFramed{\advance\hsize-\width\FrameRestore}%
	\noindent\hspace{-4.55pt}% disable indenting first paragraph
	\begin{adjustwidth}{}{7pt}%
		\vspace{2pt}\vspace{2pt}%
	}
	{%
		\vspace{2pt}\end{adjustwidth}\endMakeFramed%
}

\newenvironment{formalyellow}{%
	\def\FrameCommand{%
		\hspace{1pt}%
		{\color{yellow}\vrule width 2pt}%
		{\color{formalshade}\vrule width 4pt}%
		\colorbox{formalshade}%
	}%
	\MakeFramed{\advance\hsize-\width\FrameRestore}%
	\noindent\hspace{-4.55pt}% disable indenting first paragraph
	\begin{adjustwidth}{}{7pt}%
		\vspace{2pt}\vspace{2pt}%
	}
	{%
		\vspace{2pt}\end{adjustwidth}\endMakeFramed%
}

\geometry{
	top=20mm,
	bottom=25mm,
	left=20mm,
	right=20mm,
	bindingoffset=15mm,	
}

% ---
% PDF Information
% ---
\title{Manual Pengguna - Elbicare Audiometers}

% ---
% Document Content
% ---

\begin{document}
	
	\maketitle
	\renewcommand\contentsname{Daftar Isi}
	\tableofcontents
	\addcontentsline{toc}{chapter}{Daftar Isi}
	
	\renewcommand\listfigurename{Daftar Gambar}
	\listoffigures
	\addcontentsline{toc}{chapter}{Daftar Gambar}
	
	\renewcommand\listtablename{Daftar Tabel}
	\listoftables
	\addcontentsline{toc}{chapter}{Daftar Tabel}
	\newpage
	
	\renewcommand\chaptername{Bab}
	
	\chapter{Pendahuluan}
		\section{Tujuan Manual}
		Dokumen ini berisi informasi penting mengenai petunjuk pengoperasian perangkat Elbicare Audiometer secara aman. Perangkat ini merupakan perangkat elektronik yang dapat bekerja selama \textcolor{red}{bertahun-tahun} dengan pengoperasian yang cermat, sesuai dengan deskripsi pada dokumen ini. Pastikan untuk membaca dan memahami instruksi pada manual ini sebelum mengoperasikan perangkat Elbicare Audiometer. 
		
		\begin{formalred}
			\raisebox{0.325ex}{\resizebox{!}{2ex}{\danger}} \textbf{PERINGATAN}:
			Sebelum memulai mengoperasikan perangkat ini, silahkan baca, pahami, dan ikuti dengan seksama informasi pada Bab~\ref{chap:2} Informasi Keselamatan 
		\end{formalred}
		
		\section{Jenis Alat}
		\begin{tabular}{lcl}
			Merek & : & Elbicare\\
			Model & : & -\\
			No. Seri & : & -\\
		\end{tabular}
		
		\section{Informasi Produsen}
		\begin{tabular}{l}
			\textbf{PT Xirka Dama Persada}\\
			Gedung CM, Jl. Mataram I No. 9 Jakarta Timur, 13150, Indonesia\\
			Telepon +62-21-819-8700; Fax. +62-21-2850-7071\\
			Email: info@xooey.id\\
		\end{tabular}
		
		\section{Pelayanan Purna Jual}
		Elbicare Audiometer menawarkan servis purna jual atau garansi sejak perangkat dibeli. Detail mengenai informasi ini dapat diperoleh dengan menghubungi \textcolor{red}{PT. Xirka Dama Persada}.
		
		
	\newpage
	
	\chapter{Informasi Keselamatan}\label{chap:2}
		\section{Definisi}
		Dokumen Manual ini menggunakan tiga jenis indikator untuk memperjelas informasi penting berupa: peringatan, perhatian, dan juga catatan. Ketiga hal tersebut muncul dengan bentuk sebagaimana ditunjukkan pada Tabel ~\ref{tab:2.1}
		
		\begin{table}
			\centering
			\caption{Indikator dalam Buku Manual Pengguna}
			\label{tab:2.1}
			\begin{tabular}{|p{0.15\linewidth}  | p{0.7\linewidth}|}
				\hline
				Simbol & Keterangan\\
				\hline
				\hline
				\resizebox{!}{4ex}{\danger} & PERINGATAN: \linebreak 
				Mengindikasikan kondisi yang dapat membahayakan pasien atau operator perangkat\\
				\hline
				\resizebox{!}{4ex}{\faSearch} & PERHATIAN: \linebreak
				Mengindikasikan kondisi yang dapat merusak atau mengurangi masa pakai perangkat \\
				\hline
				\resizebox{!}{4ex}{\ding{46}} & CATATAN: \linebreak
				Mengindikasikan informasi yang dapat menjadikan pengoperasian menjadi lebih efisien dan optimal\\
				\hline
			\end{tabular}
		\end{table}
	
		Untuk dapat mengoperasikan perangkat secara tepat dan efisien, dan untuk mencegah insiden, dimohon untuk memberikan perhatian khusus pada bagian ~\ref{sec:2.2} Peringatan, serta semua peringatan dan perhatian yang ada di manual ini.
		
		\begin{formalblue}
			\raisebox{0.1ex}{\resizebox{!}{2.5ex}{\ding{46}}} \textbf{CATATAN}:
			Untuk bantuan lebih lanjut, hubungi perwakilan dari \textcolor{red}{PT. Xirka Dama Persada} 
		\end{formalblue}
		
		\section{Peringatan}\label{sec:2.2}
			\subsection{Peringatan Umum Mengenai Penggunaan Alat}
				\begin{formalred}
					\raisebox{0.125ex}{\resizebox{!}{2ex}{\danger}} \textbf{PERINGATAN}:
					Perangkat hanya dapat digunakan di bawah tanggung jawab dan rekomendasi dokter dan hanya boleh digunakan untuk kegunaan semestinya seperti tertulis pada Bab 3.1 Indikasi Penggunaan. 
				\end{formalred}
			
				\begin{formalred}
					\raisebox{0.125ex}{\resizebox{!}{2ex}{\danger}} \textbf{PERINGATAN}: 
					Manual ini menjelaskan cara merespon terhadap perangkat, tidak menjelaskan cara penanganan pasien. 
				\end{formalred}
			
				\begin{formalred}
					\raisebox{0.125ex}{\resizebox{!}{2ex}{\danger}} \textbf{PERINGATAN}: 
					Untuk menjamin kinerja perangkat secara optimal, pastikan baterai telah melalui proses charging.
				\end{formalred}
				
				\begin{formalred}
					\raisebox{0.125ex}{\resizebox{!}{2ex}{\danger}} \textbf{PERINGATAN}: 
					Gunakan hasil koreksi yang ditampilkan pada sertifikat kalibrasi yang dikeluarkan oleh institusi yang berwenang
				\end{formalred}
			
				\begin{formalred}
					\raisebox{0.125ex}{\resizebox{!}{2ex}{\danger}} \textbf{PERINGATAN}: 
					Jangan mulai menyalakan perangkat kecuali perangkat telah dirakit dengan benar, tidak tampak cacat fisik pada perangkat
				\end{formalred}
			
				\begin{formalred}
					\raisebox{0.125ex}{\resizebox{!}{2ex}{\danger}} \textbf{PERINGATAN}: 
					Gunakan perangkat tambahan hanya yang direkomendasikan oleh manual ini. Ada kemungkinan ketidakcocokan atau bahkan tidak aman jika menghubungkan perangkat dengan perangkat lain yang tidak direkomendasikan oleh manual ini.
				\end{formalred}
			
				\begin{formalred}
					\raisebox{0.125ex}{\resizebox{!}{2ex}{\danger}} \textbf{PERINGATAN}: 
					Untuk mengurangi risiko infeksi, cuci tangan dengan baik sebelum maupun setelah memegang perangkat atau aksesorinya.
				\end{formalred}
			
				\begin{formalred}
					\raisebox{0.125ex}{\resizebox{!}{2ex}{\danger}} \textbf{PERINGATAN}: 
					Perangkat dan aksesori yang kotor dapat menjadi sumber infeksi. Bersihkan perangkat dan aksesorinya sebelum dan setelah penggunaan, secara regular dan sistematis, serta mengikuti prosedur perawatan untuk mengurangi risiko infeksi. Cek di Bab 7, Petunjuk Pembersihan.
				\end{formalred}
			
				\begin{formalred}
					\raisebox{0.125ex}{\resizebox{!}{2ex}{\danger}} \textbf{PERINGATAN}: 
					Gunakan perangkat dengan hati-hati selama dan setelah penggunaan, terutama jika temperatur lingkungan relatif tinggi. Sebagian permukaan perangkat temperaturnya dapat sedikit meningkat, walaupun penggunaan masih dalam batas aman.
				\end{formalred}
			
				\begin{formalred}
					\raisebox{0.125ex}{\resizebox{!}{2ex}{\danger}} \textbf{PERINGATAN}: 
					Tidak dibolehkan memodifikasi peralatan ini.
				\end{formalred}
			
				\begin{formalred}
					\raisebox{0.125ex}{\resizebox{!}{2ex}{\danger}} \textbf{PERINGATAN}: 
					Peralatan ini tidak boleh dimodifikasi tanpa seizin pabrik. Apabila mendapatkan modifikasi tanpa persetujuan pabrik, tidak ada jaminan bahwa mesin dapat berfungsi dengan semestinya.
				\end{formalred}
			
				\begin{formalred}
					\raisebox{0.125ex}{\resizebox{!}{2ex}{\danger}} \textbf{PERINGATAN}: 
					Jika peralatan ini dimodifikasi, inspeksi dan pengujian yang sesuai harus dilakukan untuk memastikan peralatan aman untuk digunakan.
				\end{formalred}
			
				\begin{formalred}
					\raisebox{0.125ex}{\resizebox{!}{2ex}{\danger}} \textbf{PERINGATAN}: 
					Perangkat tidak boleh direndam di cairan apa pun. Cairan yang terdapat di permukaan perangkat harus dilap sesegera mungkin. Untuk menghindari kerusakan, cairan tidak boleh masuk ke perangkat.
				\end{formalred}
			
				\begin{formalred}
					\raisebox{0.125ex}{\resizebox{!}{2ex}{\danger}} \textbf{PERINGATAN}: 
					Untuk memastikan pengoperasian yang benar sehingga perangkat menjadi awet, pastikan perangkat beroperasi pada kondisi lingkungan yang direkomendasikan.
				\end{formalred}
			
				\begin{formalred}
					\raisebox{0.125ex}{\resizebox{!}{2ex}{\danger}} \textbf{PERINGATAN}: 
					Jangan operasikan perangkat di lingkungan magnetic resonance imaging (MRI), di bawah sinar matahari langsung, di dekat sumber panas, di dekat perangkat bedah yang memiliki High Frequency(HF) aktif, di luar ruangan, di dekat instalasi cairan tanpa proteksi, maupun di area berdebu.
				\end{formalred}
			
				\begin{formalred}
					\raisebox{0.125ex}{\resizebox{!}{2ex}{\danger}} \textbf{PERINGATAN}: 
					Pastikan sekeliling perangkat memungkinkan instalasi dan operasi perangkat secara baik
				\end{formalred}
			
				\begin{formalred}
					\raisebox{0.125ex}{\resizebox{!}{2ex}{\danger}} \textbf{PERINGATAN}: 
					Pastikan pemasangan perangkat ada di lokasi yang sulit dijangkau oleh anak-anak, hewan peliharaan, maupun hewan liar, dan sesuatu yang dapat menimbulkan bahaya dan kerusakan pada perangkat
				\end{formalred}
			
				\begin{formalred}
					\raisebox{0.125ex}{\resizebox{!}{2ex}{\danger}} \textbf{PERINGATAN}: 
					Untuk mengurangi risiko terbakar, jauhkan perangkat dari rokok yang menyala, korek api, dan sumber pembakaran lainnya.
				\end{formalred}			
				
			\subsection{Peringatan Mengenai Sistem Kelistrikan}
				\begin{formalred}
					\raisebox{0.125ex}{\resizebox{!}{2ex}{\danger}} \textbf{PERINGATAN}: 
					Pengisian ulang daya baterai secara rutin penting untuk memaksimalkan usia baterai. Jika tersimpan terlalu lama tanpa diisi ulang dayanya, usia baterai dapat berkurang.
				\end{formalred}
				
				\begin{formalred}
					\raisebox{0.125ex}{\resizebox{!}{2ex}{\danger}} \textbf{PERINGATAN}: 
					Catu daya listrik harus sesuai dengan spesifikasi perangkat. Pastikan tidak ada kabel yang terlipat maupun tertekan. Perangkat sebaiknya tidak dinyalakan ketika kabel listrik AC rusak. 
				\end{formalred}
				
				\begin{formalred}
					\raisebox{0.125ex}{\resizebox{!}{2ex}{\danger}} \textbf{PERINGATAN}: 
					Jangan mendekatkan perangkat pada api. Pembuangan limbah baterai harus sesuai dengan peraturan lokal yang berlaku.
				\end{formalred}
			
			\subsection{Peringatan Mengenai Peraturan}
				\begin{formalred}
					\raisebox{0.125ex}{\resizebox{!}{2ex}{\danger}} \textbf{PERINGATAN}: 
					Sebelum mulai proses audiometri, pastikan semua pengaturan sesuai dengan buku manual pengguna ini. Pastikan juga instalasi dilakukan dengan benar.
				\end{formalred}
				
				\begin{formalred}
					\raisebox{0.125ex}{\resizebox{!}{2ex}{\danger}} \textbf{PERINGATAN}: 
					Gunakan hasil koreksi yang ditampilkan pada sertifikat kalibrasi yang dikeluarkan oleh institusi yang berwenang
				\end{formalred}	
			
		\section{Simbol, Label, dan Penanda}
		\begin{table}
			\centering
			\caption{Simbol dan Penanda pada Alat}
			\label{tab:2.2}
			\begin{tabular}{|p{0.2\linewidth}  | p{0.2\linewidth}| p{0.4\linewidth}|}
				\hline
				Simbol & Lokasi & Deskripsi\\
				\hline
				\hline
				a & a & a\\
				\hline
				a & a & a\\
				\hline
			\end{tabular}
		\end{table}
		
	\newpage
	
	\chapter{Overview Sistem}
		\section{Indikasi Penggunaan}
		Elbicare Audiometer dimaksudkan untuk digunakan sebagai alat screening kondisi pendengaran manusia. Perangkat ini ditujukan untuk digunakan langsung oleh pasien dengan izin dokter. Berkaitan dengan penggunaan secara langsung, pasien selaku pengguna perlu membaca, memahami, dan mengikuti instruksi di manual ini sebelum menggunakan perangkat Elbicare Audiometer.
			\subsection{Target Pasien}
			Elbicare Audiometer ini ditujukan kepada orang/pasien yang dianggap memerlukan screening kondisi pendengaran oleh dokter.
			\subsection{Target Lingkungan Pengoperasian}
			Elbicare Audiometer ini ditujukan untuk digunakan di lokasi statis seperti di rumah sakit, ruangan medis, dan rumah pasien. Perangkat ini dapat bekerja di lingkungan dengan bising latar belakang maksimal senilai ... dB. Perangkat ini dapat bekerja di rentang temperatur ... dengan rentang kelembaban ... dan tekanan .... Penggunaan di luar rentang kondisi yang dianjurkan dapat mengakibatkan hasil yang tidak akurat dan/atau mengurangi usia penggunaan perangkat.
			
			\begin{formalred}
				\raisebox{0.125ex}{\resizebox{!}{2ex}{\danger}} \textbf{PERINGATAN}: 
				Jangan gunakan Elbicare Audiometer pada kondisi lingkungan berikut.
				\begin{itemize}
					\item Area magnetic resonance imaging (MRI)
					\item Terhubung dengan gas maupun cairan yang mudah terbakar, terutama yang dapat tercampur dengan udara, oksigen, atau nitrogen oksida
					\item Area yang rawan terhadap bencana ledakan
					\item Area yang banyak bahan peledak
					\item Ruangan tanpa ventilasi cukup
					\item Area di bawah sinar matahari langsung
					\item Area penuh debu
				\end{itemize}
				
			\end{formalred}
			
			\subsection{Target Operator}
			Elbicare Audiometer dapat dioperasikan secara langsung oleh pasien dengan izin dari dokter. Merupakan tanggung jawab klinis untuk memastikan pasien selaku pengguna (operator) memahami topik terkait mengenai penggunaan perangkat.
			\begin{formalred}
				\raisebox{0.125ex}{\resizebox{!}{2ex}{\danger}} \textbf{PERINGATAN}: 
				Elbicare Audiometer dapat digunakan dengan izin dari dokter
			\end{formalred}	
		
		\section{Indikasi Risiko}
		Sejumlah kondisi yang menjadi kontra indikasi pemakaian Audiometer adalah sebagai berikut.
		\begin{itemize}
			\item 
			\item 
			\item 
			\item 
			\item 
			\item 
			\item 
		\end{itemize}
	
		Dalam pemakaian alat secara benar, tidak ada ada kontraindikasi selain risiko yang tercantum pada Informasi Keselamatan. Segala pengaturan pada alat merupakan tanggung jawab operator. Operator perlu mempertimbangkan kondisi kesehatan pasien secara umum dalam mengoperasikan alat.
		
		Dalam pembuangan, peralatan mesin dan aksesoris yang akan dibuang termasuk dalam kategori limbah B3. Peralatan mesin maupun aksesorinya harus dibuang sesuai dengan peraturan perundangan yang berlaku di Indonesia. Tata cara pembuangannya dilakukan sesuai dengan Peraturan Menteri Kesehatan Republik Indonesia.
		
		\section{Mode Operasi}
		
		\section{Klasifikasi Perangkat}
		
		\section{Panel Depan}
		
		\section{Layar Tampilan}
		
		\section{Aksesori dan Material}
		
		\section{Pembuangan Limbah dan Aksesori}
		
	\newpage
	
	\chapter{Troubleshooting}
		\section{Overview}
		\section{Troubleshooting}
	\newpage
	
	\chapter{Petunjuk Instalasi}
		\section{Prosedur Instalasi Perangkat}
	\newpage
	
	\chapter{Petunjuk Pengoperasian}
		\section{Menyalakan Perangkat}
		\begin{enumerate}
			\item Melakukan instalasi perangkat sesuai petunjuk
			\item Menghubungkan headphone dengan Elbicare Audiometer
			\item Menekan tombol POWER di panel depan
		\end{enumerate}
	
		\section{Menjalankan Perangkat}
		\begin{enumerate}
			\item Memastikan kondisi perangkat dalam mode IDLE, menunjukkan slot SD Card sudah terhubung dengan perangkat
			\item Menekan tombol A, B, dan C secara berurutan 
			\item Proses audiometri secara otomatis akan berjalan, perangkat berada dalam mode RUN 
			\item Pasien menekan tombol A atau B atau C, dimana lampu led pada panel menyala diikuti dengan suara TONE
			\item Pasien memilih secara acak antara tombol A atau B atau C ketika tidak mendengar suara apapun
			\item Jawaban benar akan ditunjukkan dengan lampu led HIJAU menyala, sedangkan jawaban salah ditunjukkan dengan lampu led MERAH menyala
			\item Proses audiometri akan dimulai dengan melakukan screening pada telinga kiri, dilanjutkan dengan telinga kanan
			\item Pada setiap telinga, pasien akan diuji dengan TONE dengan frekuensi 250, 500, 1000, 2000, 4000, dan 8000 Hz, terdiri dari 11 level suara yang dikeluarkan
			\item Perangkat kembali ke mode IDLE setelah proses audiometri selesai
		\end{enumerate}
		
		\section{Mematikan Perangkat}
		\begin{enumerate}
			\item Memastikan proses audiometri tidak sedang berjalan/sudah selesai
			\item Menekan tombol POWER di panel depan
		\end{enumerate}
	
		\section{Pengumpulan Data}
		\begin{enumerate}
			\item Matikan perangkat Elbicare Audiometer
			\item Mencabut SD Card dari perangkat 
			\item Mengakses/mengambil data dalam SD Card dengan laptop/komputer menggunakan perangkat Card Reader 
		\end{enumerate}
	
	\newpage
	
	\chapter{Petunjuk Pembersihan, Sterilisasi, dan Disinfeksi}
		\section{Permukaan Perangkat}
		Penting untuk membersihkan bagian luar dan permukaan perangkat sebelum dan setelah penggunaan setiap pasien dan sesering mungkin. Pembersihan minimal dilakukan mingguan, atau kapan pun perangkat terlihat kotor.
		
		Untuk membersihkan permukaan perangkat:
		\begin{enumerate}
			\item Mencelupkan kain yang bersih dan lembut ke dalam campuran sabun dan air atau larutan pembersih lain seperti table di bawah.
			\item Memeras kain untuk mengurangi kelebihan cairan.
			\item Mengusap secara perlahan bagian luar dan permukaan perangkat. Diperlukan kehati- hatian agar tidak ada cairan yang masuk ke dalam rongga permukaan.
			\item Mengeringkan permukaan perangkat dengan kain bebas serat yang kering, bersih, dan lembut.
		\end{enumerate}
		
		\begin{table}
			\centering
			\caption{Larutan Pembersih yang Dapat Digunakan untuk Permukaan Perangkat}
			\label{tab:7.1}
			\begin{tabular}{|p{0.05\linewidth}  | p{0.6\linewidth}|}
				\hline
				No. & Deskripsi \\
				\hline
				\hline
				1 & Mild dishwashing detergent\\
				\hline
				2 & 70\% isopropyl alcohol (rubbing alcohol)\\
				\hline
				3 & 10\% chlorine bleach (90\% tap water)\\
				\hline
				4 & Glutaraldehyde\\
				\hline
				5 & Hospital disinfectant cleaners\\
				\hline
				6 & Hydrogen peroxide\\
				\hline
				7 & Ammonia-based household cleaners\\
				\hline
				8 & Household cleaners\\
				\hline
				9 & 15\% ammonia (85\% tap water)\\
				\hline
			\end{tabular}
		\end{table}
		
		\section{Penanganan Aksesoris}
		Pembersihan aksesoris perlu merujuk pada petunjuk yang diberikan oleh manufaktur aksesoris.
	\newpage
	
	\chapter{Petunjuk Pemeliharaan}
	\begin{formalblue}
		\raisebox{0.1ex}{\resizebox{!}{2.5ex}{\ding{46}}} \textbf{CATATAN}:
		Perangkat hanya boleh dibuka, direparasi, atau diservis oleh pihak yang ditunjuk secara resmi oleh \textcolor{red}{PT. Xirka Dama Persada} 
	\end{formalblue}

	\begin{table}
		\centering
		\caption{Jadwal Pemeliharaan Berkala}
		\label{tab:7.2}
		\begin{tabular}{|p{0.2\linewidth}  | p{0.2\linewidth}| p{0.4\linewidth}|}
			\hline
			Bagian & Pemeliharaan & Berkala\\
			\hline
			\hline
			Bagian panel depan dan panel belakang (casing) Elbicare Audiometer & Pembersihan area permukaan & Sesering mungkin, minimal mingguan atau setiap berganti pasien\\
			\hline
			a & a & a\\
			\hline
			a & a & a\\
			\hline

		\end{tabular}
	\end{table}

	\newpage
	
	\chapter{Lampiran}
		\section{Lampiran I: Rangkuman Spesifikasi Teknis}
		\section{Lampiran II: Diagram Sistem Elbilcare Audiometer}
	\newpage
	
	
	
\end{document}