
\documentclass[a4paper,12pt,oneside,pdflatex,italian,final,twocolumn]{article}



\usepackage[utf8]{inputenc}
\usepackage{parallel}
\usepackage{siunitx}
\usepackage{booktabs}
\usepackage{fancyhdr}

\usepackage[export]{adjustbox}
\usepackage[margin=0.5in]{geometry}
\addtolength{\topmargin}{0in}

\usepackage{libertine}
\renewcommand*\familydefault{\sfdefault}  %% Only if the base font of the document is to be sans serif
\usepackage[T1]{fontenc}







\title{BSPD 2019}
\author{arsphotographika }
\date{April 2019}

\begin{document}

\pagestyle{fancy}

\lhead{Q-Tronic Engineering}
\chead {\today}
\rhead{BSPD FSG 2019 v0.3}


\onecolumn

\begin{figure}
\begin{minipage}{0.47\textwidth}
\centering
\includegraphics[width=.7\textwidth,left,]{logo.png}

\end{minipage}
\hfill
\begin{minipage}{0.47\textwidth}
\raggedleft
\Huge \textbf{BSPD FSG 2019 v0.3}
\end{minipage}
\end{figure}


\begin{figure}
\begin{minipage}{0.47\textwidth}

\section{Overview}
    \begin{itemize}
        \item 2 input signal 0-5V
        \item Lower treshold 0,45V
        \item Upper treshold adjustable
        \item EV e CV Cars
        \item Auto reset after 10s
    \end{itemize}


\end{minipage}
\hfill
\begin{minipage}{0.47\textwidth}
\centering
\includegraphics[width=0.7\textwidth,right]{bspd_fsg2019.jpg}

\end{minipage}
\end{figure}



\section{Description}
\begin{itemize}

\item  The BSPD has the role of turning off the engine in the event of sudden braking and
acceleration of more than 10\% at the same time.
\item  It compliants the 2019 rules of FSG.
The circuit works with active low logic 5V.
\item  The BRAKE COMPARATOR checks that the pressure in the brake circuit is within an interval
(0 = error).
\item  The THROTTLE COMPARATOR checks that the throttle is open beyond a minimum
threshold but less than 10\% (0 = error).
\item The output signals of the two comparators are in NOR (1 = error).
The resistor R = 39 k\si{\ohm} and the capacitor C = 10 uF operate a time constant of 390 ms.
\item  The TIME 1 COMPARATOR determines a delay of about 500ms (496ms) in the activation of the system (0 = error).
The resistance R = 560 k\si{\ohm} and the capacitor C = 10uF operate a time constant of 5600ms.
\item  The TIME 2 COMPARATOR determines a delay of about 9900ms after which the BSPD is
deactivated (RESET CONDITION) and the engine can be restarted (0 = error).
The NO RELAY performs the coupling with the car's powerline.
\item Both the brake signal and the accelerator signal are compared with two
thresholds, one upper and one lower.
\item Lower threshold: 0.45V set via resistors R102 / R116 (100 \si{\ohm}, pull-up) and R112 / R121 (10 k\si{\ohm}, pull-down).
\item Upper threshold: adjustable from 0.02 V to 5 V Set by a potentiometer M64X203KB40 R101 / R115 (10 \si{\ohm} to 2.2 M\si{\ohm}, pull-up) and a resistor (10 k\si{\ohm}, pull-down).
\item  Diodes inversely polarized operate a protection against peaks on all the signals.
\item  Although in the board is written "BSPD v0.2", don't worry, the version produced is v0.3 avalaible on the official site of FSG.
\end{itemize}

\section{Technical specification}
\centering
\begin{tabular}{lcr}
\toprule
 & Unit & Value \\
\midrule
Power voltage & $V$ & 13.8 \\
Maximum current  through SC & $A$ & 3 \\
Dimensions & $mm*mm*mm$ & 44*36*11 (without cable) \\
Weight & $g$ & 13 \\
Temperature range & $°C$ & -20 ÷ 60\\
n° of inputs & $--$ & 2 \\
Voltage range signal input & $V$ & 0 ÷ 5 \\
\bottomrule
\end{tabular}

\raggedright

\section{Pinout}

\centering
\begin{tabular}{lcr}
\toprule
  Pin & Signal \\
\midrule
1 & Power 13.8 V  \\
2 & SC\_IN (Power line of the car)  \\
3 & SC\_OUT (Power line of the car)   \\
4 & S1 (Sensor, brake or throttle)  \\
5 & S2 (Sensor, brake or throttle) \\
6 & GND  \\
\bottomrule
\end{tabular}



\raggedright
\section{Measures}
\centering
\begin{figure} [h]
\centering
\includegraphics[width=0.7\textwidth,]{measures.png}

\end{figure}


\begin{figure}
\centering
\includegraphics[width=1\textwidth,left,]{schematic.png}

\end{figure}

\begin{figure}
\centering
\includegraphics[width=1\textwidth,left,]{board.png}

\end{figure}
\end{document}




