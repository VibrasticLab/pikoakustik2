\documentclass{article}

\usepackage[utf8]{inputenc}
\usepackage[T1]{fontenc}

\usepackage[yyyymmdd, hhmmss]{datetime}
\usepackage{graphicx}

\usepackage[english]{babel}
\usepackage{geometry}

\usepackage{adjustbox}
\usepackage{multirow}
\usepackage{hyperref}
\usepackage{titlesec}

\geometry{
	a4paper,
	left=25mm,
	right=25mm,
	top=25mm,
	bottom=25mm,
}

\hypersetup{
	colorlinks=true,
	linkcolor=blue,
	urlcolor=blue,
}

\setcounter{secnumdepth}{4}

\renewcommand{\theparagraph}{\thesubsubsection.\alph{paragraph}}

\begin{document}
	\begin{titlepage}
		\centering
		
		{
			\LARGE
			\bf
			Uji Respon Headphone Menggunakan Ear Simulator
		}
		
		\bigskip
		
		{
			\large
			\bf
			Tim Penelitian Portable Audiometer  
		}
		
		\vfill
	\end{titlepage}
	
	\newpage
	\section{Pendahuluan}
	
	
	\section{Tujuan Pengujian}
	Pengujian resppon headphone bertujuan untuk memperoleh informasi berupa:
	\begin{itemize}
		\item Karakteristik respon dari headphone, meliputi: frequency response, harmonic distortion, 
		\item Sound pressure level (SPL) dari luaran tone portable audiometer
	\end{itemize}
	
	\section{Metode Pengujian}
	\subsection{Alat dan Perangkat Pendukung}
	Pengujian ini memerlukan peralatan dan perangkat pendukung sebagai berikut:
	\begin{itemize}
		\item \textbf{Ear simulator}. Pada pengujian ini digunakan dua jenis Ear simulator, yaitu:
		\begin{enumerate}
			\item miniDSP EARS (\href{https://www.minidsp.com/images/documents/Product%20Brief-EARS.pdf}{detail produk})
			\item Free Space Pro II Binaural Microphone (\href{https://3diosound.com/products/free-space-pro-binaural-microphone}{detail produk})
		\end{enumerate}
	
		\item \textbf{Headphone} yang akan diuji, yaitu:
		\begin{enumerate}
			\item JBL Tune 500 Wire On-Ear Headphone 
			(\href{https://id.jbl.com/over-ear-headphones/JBL+TUNE500.html}{detail produk})
			\item Bose Frames Tempo Open-Ear Headphone
			(\href{https://www.bose.com/en_us/products/frames/bose-frames-tempo.html#v=bose_frames_tempo_black_us}{detail produk})
			\item IMOO Open-Ear Headphone
			(\href{https://imoostore.com/pages/imoo-ear-care-headset}{detail produk})
		\end{enumerate}
		
		\item Software \textbf{Room Equalization Wizard (REW)} (unduh \href{https://www.roomeqwizard.com/}{disini})
	\end{itemize}
	
	\subsection{Prosedur Pengujian}
	\subsubsection{Pengujian menggunakan miniDSP EARS}
	\paragraph{Setup dan Pengaturan Perangkat}
	\label{setup}
	\begin{enumerate}
		\item Atur nilai gain pada gain switch di panel depan miniDSP EARS (Gambar) mengikuti acuan pada Tabel. Pada pengujiani ini digunakan nilai gain sebesar 0 dB.
		
		
		\item Hubungkan miniDSP EARS ke laptop menggunakan sambungan kabel USB (Gambar) yang tersedia.
		
		\item Hubungkan headphone dengan laptop menggunakan sambungan kabel (untuk wired headphone) atau koneksi bluetooth (untuk wireless headphone)
		
	\end{enumerate}
	
	\paragraph{Kalibrasi Ear Simulator}
	\begin{enumerate}
		\item Unduh dokumen kalibrasi miniDSP EARS halaman \href{https://www.minidsp.com/products/acoustic-measurement/ears-headphone-jig}{unduhan}. Masukkan serial number dari perangkat (terdapat pada bagian panel bawah perangkat) untuk mengakses file unduhan.  
		
		\item Pilih file kalibrasi yang sesuai dengan kondisi/mode pengujian yang dilakukan. 
		
		\item Atur setup dan pengaturan perangkat sesuai dengan penejelasan pada bagian \ref{setup}
		
		\item Pastikan nilai sample rate dari perangkat input (miniDSP EARS) dan output (headphone) memiliki nilai yang sama. 
		\begin{itemize}
			\item Pengecekan sample rate pada input dapat dilakukan melalui {\bf Control Panel} $\rightarrow$ {\bf Hardware and Sound} $\rightarrow$ {\bf Manage Audio Devices} $\rightarrow$ {\bf Input Device} $\rightarrow$ {\bf Properties} $\rightarrow$ {\bf Advanced}.
			\item Pengecekan sample rate pada output dapat dilakukan melalui {\bf Control Panel} $\rightarrow$ {\bf Hardware and Sound} $\rightarrow$ {\bf Manage Audio Devices} $\rightarrow$ {\bf Output Device} $\rightarrow$ {\bf Properties} $\rightarrow$ {\bf Advanced}.
		\end{itemize}
		
		\item Buka software REW. Tampilan dialog akan muncul menunjukkan perangkat miniDSP EARS terbaca oleh laptop. Tekan {\bf Yes}.
		
		\item Tampilan dialog muncul menanyakan apakah file kalibrasi telah tersedia. Tekan {\bf Yes}.
		
		\item Unggah file kalibrasi yang akan digunakan, masing-masing untuk channel kiri dan kanan.
		
		\item Buka menu {\bf Preferences}. Pada tab Soundcard, pastikan nilai sample rate, output device, dan input device menunjukkan pengaturan seperti pada Gambar.
		
		\item Buka tab {\bf Cal Files}. Pastikan file kalibrasi sudah terunggah.
		
		\item Kembali ke menu utama. Buka menu {\bf Generator}. Pilih tab {\bf Tones} dan {\bf Sine}. Atur nilai frekuensi menjadi 300 Hz dan nilai RMS level sebesar -20.00 dB FS. Tekan tombol {\bf Play} pada bagian kanan bawah.
		
		\item Buka menu {\bf SPL Meter}. Atur volume pada headphone hingga mencapai nilai 84 dB. Nilai volume dapat diatur lebih tinggi, namun pastikan hal tersebut tidak melebihi batas kemampuan headphone.
		
		\item Kembali ke menu {\bf Generator}. Tekan tombol {\bf Stop} pada bagian kanan bawah. Tutup menu {\bf Generator} dan {\bf SPL Meter}. 
		
	\end{enumerate}
	
	\paragraph{Pengukuran dan Pengambilan Data}
	\subparagraph{Karateristik Respon Headphone}
	\begin{enumerate}
		\item 
	\end{enumerate}
	
	\subparagraph{Nilai SPL Luaran Tone}
	\begin{enumerate}
		\item 
	\end{enumerate}

	\subsubsection{Pengujian menggunakan Free Space II Binaural Microphone}
	\paragraph{Setup dan Pengaturan Perangkat}
	
	\paragraph{Kalibrasi Ear Simulator}
	
	\paragraph{Pengukuran dan Pengambilan Data}	
	
	
\end{document}