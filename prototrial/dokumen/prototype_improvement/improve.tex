\documentclass{book} % Definisi jenis dokumen

%%%%% Definisi paket-paket yang seharusnya digunakan %%%%%
\usepackage[utf8]{inputenc} % paket encoding input utf8
\usepackage[T1]{fontenc} % paket encoding huruf latin
\usepackage{tocbibind} % paket toc terdaftar dalam toc

%%%%% Definisi paket-paket yang digunakan sesuai kebutuhan %%%%%
\usepackage[yyyymmdd,hhmmss]{datetime} % paket tanggal-waktu
\usepackage{geometry} % paket ukuran kertas dan margin
\usepackage{graphicx} % paket grafik/gambar

%%%%% Pengaturan ukuran kertas dan margin %%%%%
\geometry{
    a4paper,
    left=10mm,
    right=10mm,
    top=15mm,
    bottom=15mm,
}

%%%%% Pengaturan perintah informasi perangkat lunak (hanya untuk GNU/Linux) %%%%%
\newcommand{\ShowOsVersion}{
    \immediate\write18{\unexpanded{foo=`uname -sro` && echo "${foo}" > tmp.tex}}
    \input{tmp}\immediate\write18{rm tmp.tex}
}

\newcommand{\ShowTexVersion}{
    \immediate\write18{\unexpanded{foo=`pdflatex -version | head -n1 | cut -d' ' -f1,2` && echo "${foo}" > tmp.tex}}
    \input{tmp}\immediate\write18{rm tmp.tex}
}

\begin{document}

    %%%%%%%%%%%%%%%%%%%%%%%%%%%%%%%%%%%%%%%%%%%%%%%%%%%%%%%%%%%%%%%%%

    \frontmatter % untuk halaman cover

    \begin{titlepage}

        \centering % untuk membuat tengah teks

        {
            \LARGE % pakai font besar
            \bf % pakai font BOLD
            Rangkuman Improvement Prototype P2 dan P3
        }

        \bigskip
        {\Large \bf Achmadi ST MT}
        \vfill % menambahkan ruang kosong vertikal

        \includegraphics[width=300pt]{images/elbicare-logo}
        \vfill

        \raggedright
        \noindent Dokumen ini ditulis dengan:\\ % tanda \\ menambahkan garis baru
        OS : \ShowOsVersion \\
        TeX : \ShowTexVersion \\
        Update: {\today} at \currenttime\\
    \end{titlepage}

    %%%%%%%%%%%%%%%%%%%%%%%%%%%%%%%%%%%%%%%%%%%%%%%%%%%%%%%%%%%%%%%%%

\end{document}
