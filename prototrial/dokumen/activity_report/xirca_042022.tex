\documentclass[table,dvipsnames]{beamer}
\mode<presentation>{
	\usetheme{Madrid}
	\setbeamercolor{title}{fg=Black,bg=Blue!15}
	\setbeamercolor{frametitle}{fg=Black,bg=Blue!15}
	\setbeamercolor{block title}{fg=Black,bg=Blue!15}
	\setbeamercolor{block}{fg=Black,bg=Blue!10}
}

\usepackage{graphicx}
\usepackage{booktabs}
\usepackage{xcolor}
\usepackage{multirow}
\usepackage{minted}
\usepackage[
type={CC},
modifier={by-sa},
version={4.0},
]{doclicense}

\definecolor{LightGray}{gray}{0.9}

\title[Xirca042022]{Activity Report: Xirca April 2022}
\author{}
\institute[VibrasticLab : \ccbysa]{
	Achmadi ST MT
}
\date{}

\begin{document}
    \begin{frame}
        \titlepage
    \end{frame}

	\section{Tempat dan Jadwal}
	\begin{frame}
		\begin{exampleblock}{Tempat dan Jadwal}
			Kegiatan berlangsung di:
			\begin{itemize}
				\item Tanggal: 18 April 2022 hingga 22 April 2022
				\item Pukul: 08:00 WIB hingga 16:00 WIB
				\item Tempat: Gedung Research and Development PT. Xirca Dama Persada Lantai 2.\\
				Kompleks Puri Syailendra.\\
				Jl. Lemah Neundeut No. Kav-30, Sukawarna, Setrasari, Kota Bandung, Jawa Barat 40164 .
			\end{itemize}
		\end{exampleblock}
	\end{frame}
	
	\section{Kelompok Pekerjaan}
	\begin{frame}
	    \begin{exampleblock}{Kelompok Pekerjaan}
			Lingkup utama pekerjaan yang dilakukan meliputi:
			\begin{itemize}
				\item Assembly 1 unit PCB untuk desain Audiometri P2.
				\item Review Sinyal dan Tone-Generation dari Class-D Audio DAC MAX98357A.
				\item Review sementara desain Audiometri P3.
				\item Upgrade unit packaging untuk Audiometri P2.
			\end{itemize}
	    \end{exampleblock}
	\end{frame}
	
	\section{Assembly PCB}
	\begin{frame}
		\subsection{Tujuan}
		\begin{exampleblock}{Assembly PCB}
			Merakit 1 unit tambahan PCB untuk desain Audiometri P2.
		\end{exampleblock}
	
		\subsection{Hasil}
		\begin{exampleblock}{Assembly PCB}
			Unit tambahan telah terakit
			\begin{center}
				\includegraphics[width=200pt]{images/assembly_hasil}
			\end{center}
		\end{exampleblock}
	\end{frame}

	
	\begin{frame}
		\subsection{Dokumentasi}
		\begin{exampleblock}{Memulai Assembly}
			\begin{center}
				\includegraphics[width=150pt]{images/assembly_unpack}
				\includegraphics[width=150pt]{images/assembly_start}
			\end{center}
		\end{exampleblock}
	\end{frame}

	\begin{frame}
		\begin{exampleblock}{Proses Soldering Manual}
			\begin{center}
				\includegraphics[width=200pt]{images/assembly_manual}
			\end{center}
		\end{exampleblock}
	\end{frame}
\end{document}
